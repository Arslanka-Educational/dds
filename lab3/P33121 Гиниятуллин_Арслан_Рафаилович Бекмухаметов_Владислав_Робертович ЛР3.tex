\documentclass{article}

\usepackage[utf8]{inputenc}
\usepackage[russian]{babel}
\usepackage[a4paper, margin=1in]{geometry}
\usepackage{graphicx}
\usepackage{amsmath}
\usepackage{wrapfig}
\usepackage{multirow}
\usepackage{mathtools}
\usepackage{pgfplots}
\usepackage{pgfplotstable}
\usepackage{setspace}
\usepackage{changepage}
\usepackage{caption}
\usepackage{csquotes}
\usepackage{hyperref}
\usepackage{listings}

\pgfplotsset{compat=1.18}
\hypersetup{
  colorlinks = true,
  linkcolor  = blue,
  filecolor  = magenta,      
  urlcolor   = darkgray,
  pdftitle   = ddb-report-2-giniiatullin-arslan-p33131,
}

\definecolor{codegreen}{rgb}{0,0.6,0}
\definecolor{codegray}{rgb}{0.5,0.5,0.5}
\definecolor{codepurple}{rgb}{0.58,0,0.82}
\definecolor{backcolour}{rgb}{0.99,0.99,0.99}

\lstdefinestyle{codestyle}{
  backgroundcolor=\color{backcolour},   
  commentstyle=\color{codegreen},
  keywordstyle=\color{magenta},
  numberstyle=\tiny\color{codegray},
  stringstyle=\color{codepurple},
  basicstyle=\ttfamily\footnotesize,
  breakatwhitespace=false,         
  breaklines=true,                 
  captionpos=b,                    
  keepspaces=true,                 
  numbers=left,                    
  numbersep=5pt,                  
  showspaces=false,                
  showstringspaces=false,
  showtabs=false,                  
  tabsize=2
}

\lstset{style=codestyle}

\begin{document}

\begin{titlepage}
    \begin{center}
        \begin{spacing}{1.4}
            \large{Университет ИТМО} \\
            \large{Факультет программной инженерии и компьютерной техники} \\
        \end{spacing}
        \vfill
        \textbf{
            \huge{Распределённые системы хранения данных.} \\
            \huge{Лабораторная работа №3.} \\
        }
    \end{center}
    \vfill
    \begin{center}
        \begin{tabular}{r l}
            Группа:        & P33121                      \\
            Студенты:       & Гиниятуллин Арслан Рафаилович \\
                            & Бекмухаметов Владислав Робертович   \\
            Преподаватель: & Афанасьев Дмитрий Борисович \\
            Вариант:       & 721 \\
        \end{tabular}
    \end{center}
    \vfill
    \begin{center}
        \begin{large}
            2024
        \end{large}
    \end{center}
\end{titlepage}

\section*{Ключевые слова}

База данных, PostgreSQL, резервное копирование, повреждение данных

\tableofcontents

\section{Цель работы}

Настроить процедуру периодического резервного копирования базы данных, сконфигурированной в ходе выполнения предыдущей лабораторной работы, а также разработать и отладить сценарии восстановления в случае сбоев.

\section{Текст задания}

\subsection{Описание}

\subsubsection{Этап 1. Резервное копирование}
        \begin{itemize}
            \item Настроить резервное копирование с основного узла на резервный следующим образом: Первоначальная полная копия + непрерывное архивирование. Включить для СУБД режим архивирования WAL; настроить копирование WAL (scp) на резервный узел; создать первоначальную резервную копию (pg\_basebackup), скопировать на резервный узел (rsync).
            \item Подсчитать, каков будет объем резервных копий спустя месяц работы системы, исходя из следующих условий:
                \begin{itemize}
                    \item Средний объем новых данных в БД за сутки: 850МБ.
                    \item Средний объем измененных данных за сутки: 600МБ.
                \end{itemize}
            \item Проанализировать результаты.
        \end{itemize}
\subsubsection{Этап 2. Потеря основного узла}
Этот сценарий подразумевает полную недоступность основного узла. Необходимо восстановить работу СУБД на РЕЗЕРВНОМ узле, продемонстрировать успешный запуск СУБД и доступность данных.

\subsubsection{Этап 3. Повреждение файлов БД}
Этот сценарий подразумевает потерю данных (например, в результате сбоя диска или файловой системы) при сохранении доступности основного узла. Необходимо выполнить полное восстановление данных из резервной копии и перезапустить СУБД на ОСНОВНОМ узле.

Ход работы:
    \begin{itemize}
        \item Симулировать сбой:
            \begin{itemize}
                \item удалить с диска директорию конфигурационных файлов СУБД со всем содержимым.
            \end{itemize}
        \item Проверить работу СУБД, доступность данных, перезапустить СУБД, проанализировать результаты.
        \item Выполнить восстановление данных из резервной копии, учитывая следующее условие:
            \begin{itemize}
                \item исходное расположение директории PGDATA недоступно - разместить данные в другой директории и скорректировать конфигурацию.
            \end{itemize}
        \item Запустить СУБД, проверить работу и доступность данных, проанализировать результаты.
    \end{itemize}
\subsubsection{Этап 4. Логическое повреждение данных}
Этот сценарий подразумевает частичную потерю данных (в результате нежелательной или ошибочной операции) при сохранении доступности основного узла. Необходимо выполнить восстановление данных на ОСНОВНОМ узле следующим способом:
    \begin{itemize}
        \item Генерация файла на резервном узле с помощью pg\_dump и последующее применение файла на основном узле.
    \end{itemize}
Ход работы:
\begin{itemize}
    \item В каждую таблицу базы добавить 2-3 новые строки, зафиксировать результат.
    \item Зафиксировать время и симулировать ошибку:
        \begin{itemize}
            \item удалить любые две таблицы (DROP TABLE)
        \end{itemize}
    \item Продемонстрировать результат.
    \item Выполнить восстановление данных указанным способом.
    \item Продемонстрировать и проанализировать результат.
\end{itemize}
\section{Этапы выполнения работы}

Основной кластер: pg105
Резервный кластер: pg121

\subsection{Этап 1}
Базовая настройка основного кластера
\begin{verbatim}
    echo "wal_level = replica" >> postgresql.conf
    echo "archive_mode = on" >> postgresql.conf
    echo "archive_command = 'scp %p postgres1@pg121:/var/db/postgres1/u02/gsd65/%f'" 
    >> u01/gsd65/postgresql.conf
<<<<<<< HEAD
\end{verbatim}
=======
\end{verbatim} \\
\begin{itemize}
    \item {\textit{wal\_level = replica} устанавливает уровень записи WAL на лидере, достаточный для резервного копирования с другого узла}
    \item {\textit{archive\_mode = on} указывает, что полные сегменты WAL передаются в хранилище архива командой \textit{archive\_command}}
    \item {\textit{archive\_command = 'scp \%p postgres1@pg121:/var/db/postgres1/u02/gsd65/\%f'} определяет команду, которая будет выполняться над завершенными WAL сегментами для архивации.
    (Любое вхождение \%p в этой строке заменяется путём архивируемого файла, а вхождение \%f заменяется только его именем.)
    }
\end{itemize}
\\
>>>>>>> ca4c9af (Added step 4, fixed 1)
Сохранение ssh ключа на резервный кластер.
\begin{verbatim}
    ssh-keygen -t rsa
    ssh-copy-id -i ~/.ssh/id_rsa.pub postgres1@pg121
\end{verbatim}
Создание базового бекапа и настройка rsync для директории с файлами бекапа.

\begin{verbatim}
<<<<<<< HEAD
    pg_ctl -D /var/db/postgres0/u07/gsd25 -l logfile start
    pg_basebackup -P -Ft -p 9007 -X fetch -D backups
    rsync --archive --verbose --progress --rsync-path=/usr/local/bin/rsync \
    /var/db/postgres0/backups/ postgres1@pg121:/var/db/postgres1/backups 
\end{verbatim}
=======
    pg_ctl -D /var/db/postgres0/u01/gsd65 -l logfile start
    pg_basebackup -P -Ft -p 9006 -X fetch -D backups
    rsync --archive --verbose --progress --rsync-path=/usr/local/bin/rsync \
    /var/db/postgres0/backups/ postgres1@pg121:/var/db/postgres1/backups 
\end{verbatim} \\
Подсчитаем, каков будет объем резервных копий спустя месяц работы системы, исходя из следующих условий:
    \begin{itemize}
        \item Средний объем новых данных в БД за сутки: 850МБ.
        \item Средний объем измененных данных за сутки: 600МБ.
    \end{itemize} \\
Для начала посчитаем объем бекапа
\begin{verbatim}
    [postgres0@pg105 ~/backups]$ du -sh
    7,4M	.
\end{verbatim} \\
current\_backup\_data\_volume = 7.4МБ \\ \\
Далее экстраполируем средний объём новых данных за месяц \\ \\
total\_new\_data\_volume = 850МБ * 30 = 22.5ГБ \\
total\_changed\_data\_volume = 600МБ * 30 = 18ГБ \\
total\_backup\_data\_volume = current\_backup\_data\_volume + total\_new\_data\_volume \\ + total\_changed\_data\_volume = 40507,4МБ
>>>>>>> ca4c9af (Added step 4, fixed 1)

\subsection{Этап 2}
Восстановление работы СУБД на резервном узле
\begin{verbatim}
    tar xvf backups/base.tar -C $HOME/u01/gsd65
    tar xvf backups/16384.tar -C  $HOME/u03/gsd65
    tar xvf backups/16385.tar -C $HOME/u04/gsd65
    tar xvf backups/16386.tar -C $HOME/u05/gsd65
\end{verbatim}

\begin{verbatim}
    ln -s u03/gsd65 u01/gsd65/pg_tblspc/16384
    ln -s u04/gsd65 u01/gsd65/pg_tblspc/16385
    ln -s u05/gsd65 u01/gsd65/pg_tblspc/16386
\end{verbatim}
<<<<<<< HEAD
Настраиваем restore\_command в postgresql.conf и расширяем памминг пользоваталей, так как пользователь изменился с postgres0 -> postgres1
=======
Настраиваем restore\_command в postgresql.conf и расширяем маппинг пользоваталей, так как пользователь изменился с postgres0 -> postgres1
>>>>>>> ca4c9af (Added step 4, fixed 1)

\begin{verbatim}
    echo  "restore_command = 'cp /var/db/postgres1/u02/gsd65/%f %p'" 
    >> $HOME/u01/gsd65/postgresql.conf
    echo "postgres0 postgres1 s000000" >> $HOME/u01/gsd65/pg_ident.conf 
\end{verbatim}
<<<<<<< HEAD

=======
>>>>>>> ca4c9af (Added step 4, fixed 1)
Создаем standby.signal для запуска СУБД в режиме standby
\begin{verbatim}
    touch u01/gsd65/standy.signal
\end{verbatim}
<<<<<<< HEAD

=======
>>>>>>> ca4c9af (Added step 4, fixed 1)
Запускаем
\begin{verbatim}
    pg_ctl -D u01/gsd65 -l файл_журнала start
\end{verbatim}

\subsection{Этап 3}
<<<<<<< HEAD
Повреждение файлов бд и полное восстановление данных на основном узле с резервного. 

=======
Повреждение файлов бд и полное восстановление данных на основном узле с резервного. \\
>>>>>>> ca4c9af (Added step 4, fixed 1)
Симулируем сбой: удаляем с диска директорию конфигурационных файлов
\begin{verbatim}
    rm -fr ~/u01/gsd65/*
    mkdir ~/u10/gsd65
    chmod 700 ~/u10/gsd65
<<<<<<< HEAD
\end{verbatim}

=======
\end{verbatim} \\
>>>>>>> ca4c9af (Added step 4, fixed 1)
Подтягиваем данные с резервного узла для восстановления
\begin{verbatim}
    mkdir ~/new_backups
    scp postgres1@pg121:/var/db/postgres1/backups/base.tar ~/new_backups/
    scp postgres1@pg121:/var/db/postgres1/backups/16384.tar ~/new_backups/
    scp postgres1@pg121:/var/db/postgres1/backups/16385.tar ~/new_backups/
    scp postgres1@pg121:/var/db/postgres1/backups/16386.tar ~/new_backups/
<<<<<<< HEAD
\end{verbatim}

=======
\end{verbatim} \\
>>>>>>> ca4c9af (Added step 4, fixed 1)
Восстанавливаем данные в других директориях
\begin{verbatim}
    tar xvf ~/new_backups/base.tar -C ~/u10/gsd65
    tar xvf ~/new_backups/16384.tar -C ~/u03/gsd65
    tar xvf ~/new_backups/16385.tar -C ~/u04/gsd65
    tar xvf ~/new_backups/16386.tar -C ~/u05/gsd65
<<<<<<< HEAD
\end{verbatim}
=======
\end{verbatim} \\
>>>>>>> ca4c9af (Added step 4, fixed 1)
Задаем команду для восстановления, создаем сигнал для запуска в recovery mod и запускаем сервер
\begin{verbatim}
    echo "restore_command = 'scp postgres1@pg121:/var/db/postgres1/u02/gsd65/%f %p'" >> $HOME/u10/gsd65/postgresql.conf
    touch $HOME/u10/gsd65/recovery.signal
    pg_ctl -D $HOME/u10/gsd65 -l logfile start
    echo "postgres0 postgres1 s000000" >> $HOME/u01/gsd65/pg_ident.conf 
\end{verbatim}

\subsection{Этап 4}

<<<<<<< HEAD
Мы крутые все сделали

\begin{verbatim}
greatercapybara=> select now();
              now
-------------------------------
 2024-04-29 15:02:50.933467+03
(1 строка)

Создадим dump на резервном сервере

mkdir ~/dumps
pg_dump  -p 9007 -d greatercapybara -U postgres0 -Fc > ~/dumps/db-$(date +"%m-%d-%Y-%H-%M-%S").dump

Отправим его на основной

rsync --archive ~/dumps postgres0@pg106:
\end{verbatim}
    greatercapybara=> INSERT INTO tst values(123);
    greatercapybara=> INSERT INTO tst values(111);
    greatercapybara=> INSERT INTO tst values(222);
    greatercapybara=> select * from tst;
     id
    -----
     123
     111
     222
    (3 строки)
=======
Вставим строки на основном узле

\begin{verbatim}
    greatercapybara=# INSERT INTO wolf (name, type, age) VALUES ('name_1', 'default', 1);
    INSERT 0 1
    greatercapybara=# INSERT INTO wolf (name, type, age) VALUES ('name_2', 'default', 2);
    INSERT 0 1
    greatercapybara=# INSERT INTO wolf (name, type, age) VALUES ('name_3', 'default', 3);
    INSERT 0 1
    greatercapybara=# select * from wolf;
     id |  name  |  type   | age
    ----+--------+---------+-----
      1 | name_1 | default |   1
      2 | name_2 | default |   2
      3 | name_3 | default |   3
    (3 строки)
\end{verbatim} \\
Проверим их наличие на резервном узле

\begin{verbatim}
    greatercapybara=# select * from wolf;
     id |  name  |  type   | age
    ----+--------+---------+-----
      1 | name_1 | default |   1
      2 | name_2 | default |   2
      3 | name_3 | default |   3
    (3 строки)
\end{verbatim} \\
Фиксируем время

\begin{verbatim}
    greatercapybara=# select now();
                now
    -------------------------------
     2024-04-28 20:15:12.259434+03
    (1 строка)
\end{verbatim} \\
Создадим dump на резервном сервере

\begin{verbatim}
    $ mkdir ~/dumps
    $ pg_dump  -p 9006 -d greatercapybara -U postgres0 -Fc > ~/dumps/db-$(date +"%m-%d-%Y-%H-%M-%S").dump
\end{verbatim} \\
Отправим его на основной

\begin{verbatim}
    $ rsync --archive ~/dumps postgres0@pg105:
\end{verbatim} \\
Правим данные

\begin{verbatim}
    greatercapybara=# INSERT INTO wolf (name, type, age) VALUES ('corrupted_name', 'default', 4);
    INSERT 0 1
    greatercapybara=# INSERT INTO wolf (name, type, age) VALUES ('corrupted_name_2', 'default', 4);
    INSERT 0 1
    greatercapybara=# select * from wolf;
     id |       name       |  type   | age
    ----+------------------+---------+-----
      1 | name_1           | default |   1
      2 | name_2           | default |   2
      3 | name_3           | default |   3
      4 | corrupted_name   | default |   4
      5 | corrupted_name_2 | default |   4
    (5 строк)
\end{verbatim} \\
Восстановимся на основном сервере из dump-а

\begin{verbatim}
    $ pg_restore -p 9067 -d yellowfox -c db-04-28-2024-20-34-55.dump
\end{verbatim} \\
Проверим данные

\begin{verbatim}
    greatercapybara=# select * from wolf;
     id |  name  |  type   | age
    ----+--------+---------+-----
      1 | name_1 | default |   1
      2 | name_2 | default |   2
      3 | name_3 | default |   3
    (3 строки)
    
    greatercapybara=# select now();
                  now
    -------------------------------
     2024-04-28 20:38:42.713888+03
    (1 строка)
\end{verbatim}

>>>>>>> ca4c9af (Added step 4, fixed 1)
\end{document}