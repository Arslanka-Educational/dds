\documentclass{article}

\usepackage[utf8]{inputenc}
\usepackage[russian]{babel}
\usepackage[a4paper, margin=1in]{geometry}
\usepackage{graphicx}
\usepackage{amsmath}
\usepackage{wrapfig}
\usepackage{multirow}
\usepackage{mathtools}
\usepackage{pgfplots}
\usepackage{pgfplotstable}
\usepackage{setspace}
\usepackage{changepage}
\usepackage{caption}
\usepackage{csquotes}
\usepackage{hyperref}
\usepackage{listings}

\pgfplotsset{compat=1.18}
\hypersetup{
  colorlinks = true,
  linkcolor  = blue,
  filecolor  = magenta,      
  urlcolor   = darkgray,
  pdftitle   = ddb-report-2-giniiatullin-arslan-p33131,
}

\definecolor{codegreen}{rgb}{0,0.6,0}
\definecolor{codegray}{rgb}{0.5,0.5,0.5}
\definecolor{codepurple}{rgb}{0.58,0,0.82}
\definecolor{backcolour}{rgb}{0.99,0.99,0.99}

\lstdefinestyle{codestyle}{
  backgroundcolor=\color{backcolour},   
  commentstyle=\color{codegreen},
  keywordstyle=\color{magenta},
  numberstyle=\tiny\color{codegray},
  stringstyle=\color{codepurple},
  basicstyle=\ttfamily\footnotesize,
  breakatwhitespace=false,         
  breaklines=true,                 
  captionpos=b,                    
  keepspaces=true,                 
  numbers=left,                    
  numbersep=5pt,                  
  showspaces=false,                
  showstringspaces=false,
  showtabs=false,                  
  tabsize=2
}

\lstset{style=codestyle}

\begin{document}

\begin{titlepage}
    \begin{center}
        \begin{spacing}{1.4}
            \large{Университет ИТМО} \\
            \large{Факультет программной инженерии и компьютерной техники} \\
        \end{spacing}
        \vfill
        \textbf{
            \huge{Распределённые системы хранения данных.} \\
            \huge{Лабораторная работа №3.} \\
        }
    \end{center}
    \vfill
    \begin{center}
        \begin{tabular}{r l}
            Группа:        & P33121                      \\
            Студенты:       & Гиниятуллин Арслан Рафаилович \\
                            & Бекмухаметов Владислав Робертович   \\
            Преподаватель: & Афанасьев Дмитрий Борисович \\
            Вариант:       & 721 \\
        \end{tabular}
    \end{center}
    \vfill
    \begin{center}
        \begin{large}
            2024
        \end{large}
    \end{center}
\end{titlepage}

\section*{Ключевые слова}

База данных, PostgreSQL, резервное копирование, повреждение данных

\tableofcontents

\section{Цель работы}

Настроить процедуру периодического резервного копирования базы данных, сконфигурированной в ходе выполнения предыдущей лабораторной работы, а также разработать и отладить сценарии восстановления в случае сбоев.

\section{Текст задания}

\subsection{Описание}

\subsubsection{Этап 1. Резервное копирование}
        \begin{itemize}
            \item Настроить резервное копирование с основного узла на резервный следующим образом: Первоначальная полная копия + непрерывное архивирование. Включить для СУБД режим архивирования WAL; настроить копирование WAL (scp) на резервный узел; создать первоначальную резервную копию (pg\_basebackup), скопировать на резервный узел (rsync).
            \item Подсчитать, каков будет объем резервных копий спустя месяц работы системы, исходя из следующих условий:
                \begin{itemize}
                    \item Средний объем новых данных в БД за сутки: 850МБ.
                    \item Средний объем измененных данных за сутки: 600МБ.
                \end{itemize}
            \item Проанализировать результаты.
        \end{itemize}
\subsubsection{Этап 2. Потеря основного узла}
Этот сценарий подразумевает полную недоступность основного узла. Необходимо восстановить работу СУБД на РЕЗЕРВНОМ узле, продемонстрировать успешный запуск СУБД и доступность данных.

\subsubsection{Этап 3. Повреждение файлов БД}
Этот сценарий подразумевает потерю данных (например, в результате сбоя диска или файловой системы) при сохранении доступности основного узла. Необходимо выполнить полное восстановление данных из резервной копии и перезапустить СУБД на ОСНОВНОМ узле.

Ход работы:
    \begin{itemize}
        \item Симулировать сбой:
            \begin{itemize}
                \item удалить с диска директорию конфигурационных файлов СУБД со всем содержимым.
            \end{itemize}
        \item Проверить работу СУБД, доступность данных, перезапустить СУБД, проанализировать результаты.
        \item Выполнить восстановление данных из резервной копии, учитывая следующее условие:
            \begin{itemize}
                \item исходное расположение директории PGDATA недоступно - разместить данные в другой директории и скорректировать конфигурацию.
            \end{itemize}
        \item Запустить СУБД, проверить работу и доступность данных, проанализировать результаты.
    \end{itemize}
\subsubsection{Этап 4. Логическое повреждение данных}
Этот сценарий подразумевает частичную потерю данных (в результате нежелательной или ошибочной операции) при сохранении доступности основного узла. Необходимо выполнить восстановление данных на ОСНОВНОМ узле следующим способом:
    \begin{itemize}
        \item Генерация файла на резервном узле с помощью pg\_dump и последующее применение файла на основном узле.
    \end{itemize}
Ход работы:
\begin{itemize}
    \item В каждую таблицу базы добавить 2-3 новые строки, зафиксировать результат.
    \item Зафиксировать время и симулировать ошибку:
        \begin{itemize}
            \item удалить любые две таблицы (DROP TABLE)
        \end{itemize}
    \item Продемонстрировать результат.
    \item Выполнить восстановление данных указанным способом.
    \item Продемонстрировать и проанализировать результат.
\end{itemize}
\subsection{Этапы выполнения работы}


\end{document}