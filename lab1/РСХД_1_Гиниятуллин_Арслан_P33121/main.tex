\documentclass{article}

\usepackage[utf8]{inputenc}
\usepackage[russian]{babel}
\usepackage[a4paper, margin=1in]{geometry}
\usepackage{graphicx}
\usepackage{amsmath}
\usepackage{wrapfig}
\usepackage{multirow}
\usepackage{mathtools}
\usepackage{pgfplots}
\usepackage{pgfplotstable}
\usepackage{setspace}
\usepackage{changepage}
\usepackage{caption}
\usepackage{csquotes}
\usepackage{hyperref}
\usepackage{listings}

\pgfplotsset{compat=1.18}
\hypersetup{
  colorlinks = true,
  linkcolor  = blue,
  filecolor  = magenta,      
  urlcolor   = darkgray,
  pdftitle   = ddb-report-1-smirnov-victor-p33131,
}

\definecolor{codegreen}{rgb}{0,0.6,0}
\definecolor{codegray}{rgb}{0.5,0.5,0.5}
\definecolor{codepurple}{rgb}{0.58,0,0.82}
\definecolor{backcolour}{rgb}{0.99,0.99,0.99}

\lstdefinestyle{codestyle}{
  backgroundcolor=\color{backcolour},   
  commentstyle=\color{codegreen},
  keywordstyle=\color{magenta},
  numberstyle=\tiny\color{codegray},
  stringstyle=\color{codepurple},
  basicstyle=\ttfamily\footnotesize,
  breakatwhitespace=false,         
  breaklines=true,                 
  captionpos=b,                    
  keepspaces=true,                 
  numbers=left,                    
  numbersep=5pt,                  
  showspaces=false,                
  showstringspaces=false,
  showtabs=false,                  
  tabsize=2
}

\lstset{style=codestyle}

\begin{document}

\begin{titlepage}
    \begin{center}
        \begin{spacing}{1.4}
            \large{Университет ИТМО} \\
            \large{Факультет программной инженерии и компьютерной техники} \\
        \end{spacing}
        \vfill
        \textbf{
            \huge{Распределённые системы хранения данных.} \\
            \huge{Лабораторная работа №1.} \\
        }
    \end{center}
    \vfill
    \begin{center}
        \begin{tabular}{r l}
            Группа:        & P33121                      \\
            Студент:       & Гиниятуллин Арслан Рафаилович     \\
            Преподаватель: & Афанасьев Дмитрий Борисович \\
            Вариант:       & 661                         \\
        \end{tabular}
    \end{center}
    \vfill
    \begin{center}
        \begin{large}
            2024
        \end{large}
    \end{center}
\end{titlepage}

\section*{Ключевые слова}

База данных, PostgreSQL, системный каталог.

\tableofcontents

\section{Цель работы}

Научиться проектировать базы данных, составлять инфологические и даталогические модели данных, реализовывать их в БД PostgreSQL, научиться выполнять запросы.

\section{Текст задания}

ИИспользуя сведения из системных каталогов, сформировать запрос, реализующий полнотекстовый поиск по исходному коду всех процедур, функций и триггеров в пределах заданной схемы. Поиск должен осуществляться независимо от регистра символов в строке запроса. Программу оформить в виде процедуры. \\

Пример вывода:
\begin{verbatim}
Текст запроса: Н_Люди

 No. Имя объекта	          # строки	     Текст
 --- -------------------   -------------  --------------------------------------------
   1 MyFunction1           16		            SELECT * FROM н_люди WHERE
   2 MyProcedure1	         42		            INSERT INTO Н_ЛЮДИ
		...
\end{verbatim}

\section{Инициализация объектов}

Далее был написан SQL скрипт, создающий таблицу, аналогичную той, что в примере.

\lstinputlisting[language=SQL]{init.sql}

\section{Реализация скрипта}

\lstinputlisting[language=SQL]{find_fpt_declarations_like.sql}

\section{Вывод скрипта}

\begin{verbatim}
No.  Имя объекта          # строки       Текст
---  -------------------- -------------  --------------------------------------------
1    auditlog              1                  INSERT INTO AUDIT(EMP_ID, ENTRY_DATE) VAL
2    auditlog              10             CREATE OR REPLACE FUNCTION public.auditlog()
3    example_trigger       1              CREATE TRIGGER example_trigger AFTER INSERT O
    
\end{verbatim}
\section{Вывод}

Данная лабораторная работа помогла мне изучить системный каталог PostgreSQL.

\end{document}