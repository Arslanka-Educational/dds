\documentclass{article}

\usepackage[utf8]{inputenc}
\usepackage[russian]{babel}
\usepackage[a4paper, margin=1in]{geometry}
\usepackage{graphicx}
\usepackage{amsmath}
\usepackage{wrapfig}
\usepackage{multirow}
\usepackage{mathtools}
\usepackage{pgfplots}
\usepackage{pgfplotstable}
\usepackage{setspace}
\usepackage{changepage}
\usepackage{caption}
\usepackage{csquotes}
\usepackage{hyperref}
\usepackage{listings}

\pgfplotsset{compat=1.18}
\hypersetup{
  colorlinks = true,
  linkcolor  = blue,
  filecolor  = magenta,      
  urlcolor   = darkgray,
  pdftitle   = ddb-report-2-giniiatullin-arslan-p33131,
}

\definecolor{codegreen}{rgb}{0,0.6,0}
\definecolor{codegray}{rgb}{0.5,0.5,0.5}
\definecolor{codepurple}{rgb}{0.58,0,0.82}
\definecolor{backcolour}{rgb}{0.99,0.99,0.99}

\lstdefinestyle{codestyle}{
  backgroundcolor=\color{backcolour},   
  commentstyle=\color{codegreen},
  keywordstyle=\color{magenta},
  numberstyle=\tiny\color{codegray},
  stringstyle=\color{codepurple},
  basicstyle=\ttfamily\footnotesize,
  breakatwhitespace=false,         
  breaklines=true,                 
  captionpos=b,                    
  keepspaces=true,                 
  numbers=left,                    
  numbersep=5pt,                  
  showspaces=false,                
  showstringspaces=false,
  showtabs=false,                  
  tabsize=2
}

\lstset{style=codestyle}

\begin{document}

\begin{titlepage}
    \begin{center}
        \begin{spacing}{1.4}
            \large{Университет ИТМО} \\
            \large{Факультет программной инженерии и компьютерной техники} \\
        \end{spacing}
        \vfill
        \textbf{
            \huge{Распределённые системы хранения данных.} \\
            \huge{Лабораторная работа №2.} \\
        }
    \end{center}
    \vfill
    \begin{center}
        \begin{tabular}{r l}
            Группа:        & P33121                      \\
            Студенты:       & Гиниятуллин Арслан Рафаилович \\
                            & Бекмухаметов Владислав Робертович   \\
            Преподаватель: & Афанасьев Дмитрий Борисович \\
            Вариант:       & 6 \\
        \end{tabular}
    \end{center}
    \vfill
    \begin{center}
        \begin{large}
            2024
        \end{large}
    \end{center}
\end{titlepage}

\section*{Ключевые слова}

База данных, PostgreSQL, кластер, конфигурация.

\tableofcontents

\section{Цель работы}

Научиться создавать и конфигурировать кластер баз данных PostgreSQL, сами БД, табличные пространства и роли.

\section{Текст задания}

\subsection{Описание}
На выделенном узле создать и сконфигурировать новый кластер БД, саму БД,
табличные пространства и новую роль в соответствии с заданием. Произвести
наполнение базы.
\\ \\
Отчёт должен содержать все команды по настройке, а также измененные строки
конфигурационных файлов.

\begin{itemize}
    \item  Подключение к узлу через helios: 
            \begin{verbatim} ssh -J sXXXXXX@helios.cs.ifmo.ru:2222 postgresY@pgZZZ\end{verbatim} 
    \item С самого helios или из учебных классов: 
            \begin{verbatim}ssh postgresY@pgZZZ\end{verbatim}
\end{itemize} 
\\
Персональный пароль для работы с узлом выдается преподавателем.
Обратите внимание, что домашняя директория пользователя \textbf{/var/postgres/\$LOGNAME}.
\subsection{Этапы выполнения работы}

Этапы выполнения работы:
\begin{enumerate}
    \item Инициализация кластера БД
          \begin{itemize}
              \item Имя узла — \textbf{pg105}
              \item Имя пользователя — \textbf{postgres0}
              \item Директория кластера БД — \textbf{\$HOME/u01/gsd65}
              \item Кодировка, локаль — \textbf{KOI8-R, русская}
              \item Перечисленные параметры задать через аргументы команды.
          \end{itemize}
    \item Конфигурация и запуск сервера БД
          \begin{itemize}
              \item Способ подключения к БД — \textbf{TCP/IP socket, номер порта 9006}
              \item Остальные способы подключений \textbf{запретить}
              \item Способ аутентификации клиентов — \textbf{по имени пользователя}
              \item Настроить следующие параметры сервера: \begin{verbatim}
max_connections, shared_buffers, temp_buffers, work_mem,
checkpoint_timeout, effective_cache_size, fsync, commit_delay\end{verbatim}
               Параметры должны быть подобраны в соответствии со сценарием OLTP: \textbf{1000 транзакций/сек}. с записью размером по \textbf{8 КБ}, акцент на высокую доступность данных;
              \item Директория WAL файлов — \textbf{\$HOME/u02/gsd65}
              \item Формат лог-файлов — \textbf{csv}
              \item Уровень сообщений лога — \textbf{ERROR}
              \item Дополнительно логировать — завершение сессий и продолжительность
выполнения команд
          \end{itemize}
    \item Дополнительные табличные пространства и наполнение
          \begin{itemize}
              \item Создать новые табличные пространства для различных таблиц:
                    \begin{itemize}
                        \item \textbf{\$HOME/u03/gsd65}
                        \item \textbf{\$HOME/u04/gsd65}
                        \item \textbf{\$HOME/u05/gsd65}
                    \end{itemize}
                \item На основе \textbf{template0} создать новую базу — \textbf{greatercapybara}
            \item От имени новой роли (не администратора) произвести наполнение
существующих баз тестовыми наборами данных. Предоставить права по
необходимости. Табличные пространства должны использоваться по назначению.
            \item Вывести список всех табличных пространств кластера и содержащиеся
в них объекты.
          \end{itemize}
\end{enumerate}

\section{Выполнение этапов работы}

\subsection{Инициализация кластера БД}

\subsubsection{Подключение к узлу с helios}
Подключаемся к узлу \textbf{pg105} под пользователем \textbf{postgres0}
\begin{verbatim}
    ssh postgres0@pg105
\end{verbatim}

\subsubsection{Создание кластера}
Создадим соответствующий каталог для будущего кластера PostgreSQL, предоставим права владения пользователю \textbf{postgres0}, чтобы была возможность создание подкаталогов, перейдем под его управление.
\begin{verbatim}
    mkdir -p $HOME/u01/gsd65
    chown postgres0 $HOME/u01/gsd65
    su postgres0
\end{verbatim}
\\
Далее инициализируем кластер в ранее созданном каталоге --pgdata=\textbf{\$HOME/u01/gsd65}, c именем пользователя --username=\textbf{postgres0}, кодировкой --encoding=\textbf{KOI8R} и локалью --locale=\textbf{ru\_RU.KOI8\-R}.
\begin{verbatim}
    initdb --pgdata=$HOME/u01/gsd65 --username=postgres0 --encoding=KOI8R --locale=ru_RU.KOI8-R
\end{verbatim}

\section{Конфигурация и запуск сервера БД}

\subsection{Настройка способа подключения и аутентификации в файле pg\_hba.conf}
Для настройка аутентификации клиентов по \textbf{TCP/IP socket}, используя имя пользователя, исключая все другие способы подключения, нужно отредактировать конфигурационный файл \textbf{pg\_hba.conf}, оставив там одну раскомментированную строку (см. ниже)

\begin{verbatim}
 #TYPE  DATABASE        USER            ADDRESS                 METHOD
  host   all             all                                     ident
\end{verbatim}
\\
Здесь \textbf{host} – указывает на управление подключениями, устанавливаемыми по \textbf{TCP/IP}. \\ \\ 
\textbf{all} – указывает на то, что запись подходит всем БД и всем пользователям. \\ \\ 
\textbf{ident} – получает имя пользователя операционной системы клиента, связываясь с сервером Ident, и проверяет, соответствует ли оно имени пользователя базы данных. Аутентификация ident может использоваться только для подключений по \textbf{TCP/IP}.

\subsection{Настройка параметров сервера в файле postgresql.conf}

\subsubsection{Номер порта для подключения}
Для подключения к БД по порту \textbf{9006} добавим в конфигурационный файл \textbf{postgresql.conf} следующую строку.
\begin{verbatim}
#------------------------------------------------------------------------------
# CONNECTIONS AND AUTHENTICATION
#------------------------------------------------------------------------------

# - Connection Settings -
port = 9006
\end{verbatim}

\subsubsection{Настройка параметров сервера для OLTP сценария}
\textbf{Описание сценария}: \textbf{1000 транзакций/сек} с
записью размером по \textbf{8 КБ}, акцент на высокую доступность данных.
\\ \\
Установим значения \textbf{max\_connections}=1000 для одновременного обслуживания 1000 подключений. Так как одно соединение може обрабатывать ону транзакцию за промежуток времени.
\begin{verbatim}
    #max_connections
\end{verbatim}
\section{Дополнительные табличные пространства и наполнение}

Создаем директории для новых таблих пространств. Каталог должен быть пустым и принадлежать пользователю ОС, под которым запущен PostgreSQL.
\begin{verbatim}
    mkdir -p $HOME/u03/gsd65
    mkdir -p $HOME/u04/gsd65
    mkdir -p $HOME/u05/gsd65
\end{verbatim}
Подключаемся к нашему серверу
\begin{verbatim}
    psql -p 9006 -u postgres0 postgres
\end{verbatim}
Создаем необходимые табличные пространства. 
Создавать табличное пространство должен суперпользователь базы данных, но после этого можно разрешить обычным пользователям его использовать. После создания можно проверить успешность командой db
\begin{verbatim}
    CREATE TABLESPACE u03_gsd65 LOCATION '/var/db/postgres0/u03/gsd65';
    CREATE TABLESPACE u04_gsd65 LOCATION '/var/db/postgres0/u04/gsd65';
    CREATE TABLESPACE u05_gsd65 LOCATION '/var/db/postgres0/u05/gsd65';
    \db
\end{verbatim}
На основе template0 создаем новую базу данных **greatercapybara**
\begin{verbatim}
    createdb -p 9006 -T template1 greatercapybara
\end{verbatim}
Создаем тестовые таблицы. Для указания табличного пространства можно использовать TABLESPACE, а можно использовать параметр default\_tablespace 
\begin{verbatim}
    CREATE TABLE xxx ( id serial primary key, name text ) TABLESPACE u03_gsd65;
\end{verbatim}
\begin{verbatim}
    SET default_tablespace = u03_gsd65;
    CREATE TABLE yyy ( id serial primary key, name text, xxx_id int);
\end{verbatim}
Создаем тестового пользователя и выдаем необходимые права
\begin{verbatim}
    CREATE USER s000000 WITH PASSWORD '1111';
    GRANT CONNECT ON DATABASE greatercapybara TO s000000;
    GRANT USAGE ON SCHEMA public TO s000000;
\end{verbatim}
Наполнение данными
\begin{verbatim}
    psql -U s000000 -d greatercapybara
    
    INSERT INTO xxx (name)
    VALUES ('name1'), ('name2'), ('name3');
    
    INSERT INTO yyy (name, xxx_id)
    VALUES ('name4', 1), ('name5', 2);
\end{verbatim}
Список всех табличных пространств кластера и содержащиеся в них объекты
\begin{verbatim}
    SELECT ts.spcname AS tablespace_name, NULL AS table_name, NULL AS object_type
    FROM pg_tablespace ts
    UNION ALL
    SELECT ts.spcname AS tablespace_name, c.relname AS table_name, 'table' AS object_type
    FROM pg_tablespace ts
            JOIN pg_class c ON ts.oid = c.reltablespace
    WHERE c.relkind = 'r'
    UNION ALL
    SELECT ts.spcname AS tablespace_name, c.relname AS index_name, 'index' AS object_type
    FROM pg_tablespace ts
            JOIN pg_class c ON ts.oid = c.reltablespace
            JOIN pg_index i ON c.oid = i.indexrelid
    UNION ALL
    SELECT ts.spcname AS tablespace_name, c.relname AS sequence_name, 'sequence' AS object_type
    FROM pg_tablespace ts
            JOIN pg_class c ON ts.oid = c.reltablespace
    WHERE c.relkind = 'S'
    ORDER BY tablespace_name, table_name;
\end{verbatim}
\end{document}